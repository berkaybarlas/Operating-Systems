
\documentclass{article}

\usepackage[version=3]{mhchem} % Package for chemical equation typesetting
\usepackage{siunitx} % Provides the \SI{}{} and \si{} command for typesetting SI units
\usepackage{graphicx} % Required for the inclusion of images
\usepackage{natbib} % Required to change bibliography style to APA
\usepackage{amsmath} % Required for some math elements 
\usepackage{caption}
\usepackage{subcaption}
\usepackage{listings}
\usepackage{color}
 
\definecolor{codegreen}{rgb}{0,0.6,0}
\definecolor{codegray}{rgb}{0.5,0.5,0.5}
\definecolor{codepurple}{rgb}{0.58,0,0.82}
\definecolor{backcolour}{rgb}{0.95,0.95,0.92}
 
\lstdefinestyle{mystyle}{
    backgroundcolor=\color{backcolour},   
    commentstyle=\color{codegreen},
    keywordstyle=\color{codepurple},
    numberstyle=\tiny\color{codegray},
    stringstyle=\color{codepurple},
    basicstyle=\footnotesize,
    breakatwhitespace=false,         
    breaklines=true,                 
    captionpos=b,                    
    keepspaces=true,                 
    numbers=left,                    
    numbersep=5pt,                  
    showspaces=false,                
    showstringspaces=false,
    showtabs=false,                  
    tabsize=2
}
\lstset{style=mystyle}

\setlength\parindent{0pt} % Removes all indentation from paragraphs

\renewcommand{\labelenumi}{\alph{enumi}.} % Make numbering in the enumerate environment by letter rather than number (e.g. section 6)

\newcommand\tab[1][0.5cm]{\hspace*{#1}}

%\usepackage{times} % Uncomment to use the Times New Roman font

\renewcommand{\thesubsection}{\thesection.\alph{subsection}}

%----------------------------------------------------------------------------------------
%	DOCUMENT INFORMATION
%----------------------------------------------------------------------------------------

\title{COMP 304: Assignment 2} % Title

\author{Berkay \textsc{Barlas}} % Author name

\date{\today} % Date for the report

\begin{document}

\maketitle % Insert the title, author and date

\begin{center}
\begin{tabular}{l r}
Date Performed: & April 15, 2019 \\ % Date the experiment was performed
Instructor: & Didem Unat % Instructor/supervisor
\end{tabular}
\end{center}

% If you wish to include an abstract, uncomment the lines below
% \begin{abstract}
% Abstract text
% \end{abstract}

%----------------------------------------------------------------------------------------
%	SECTION 1
%----------------------------------------------------------------------------------------


\section{Problem 1}

\subsection{}Draw four Gantt charts illustrating the execution of these processes using FCFS, SJF,
a non- preemptive priority (a smaller priority number implies a higher priority), and RR
(quantum = 4 ms) scheduling

\begin{description}
    \item[Process Execution Gantt Chart of FCFS] \hfill 
    \\
    \\ 
    \includegraphics[width=1\linewidth]{./fcfs.png}
    \item[Process Execution Gantt Chart of SJF] \hfill 
    \\
    \\ 
    \includegraphics[width=1\linewidth]{./sjf.png}
    \item[Process Execution Gantt Chart of non-preemptive Priority] \hfill 
    \\
    \\ 
    \includegraphics[width=1\linewidth]{./priority.png}
    \item[Process Execution Gantt Chart of RR] \hfill 
    \\
    \\ 
    \includegraphics[width=1\linewidth]{./rr.png}
\end{description}

\subsection{} If the context-switch overhead is 1 ms, calculate the waiting time of each process for
each of the scheduling algorithms in part (a). Which of the schedules results in the minimal
average waiting time? 

SJF(Shortest Job First) results in the minimal average waiting time.\\ \\

\begin{description}
    \item[Process Execution Gantt Chart of FCFS] \hfill 
    \begin{description}
        \item[Waiting Time of P1:] 0
        \item[Waiting Time of P2:] 9 
        \item[Waiting Time of P3:] 15
        \item[Waiting Time of P4:] 19
        \item[Waiting Time of P5:] 21
        \item[Average waiting time:] ( 9 + 15 + 19 + 21 ) / 5 = 12,8
    \end{description}
    \item[Process Execution Gantt Chart of SJF] \hfill 
    \begin{description}
        \item[Waiting Time of P1:] 12
        \item[Waiting Time of P2:] 6 
        \item[Waiting Time of P3:] 2
        \item[Waiting Time of P4:] 0
        \item[Waiting Time of P5:] 20
        \item[Average waiting time:] ( 12 + 6 + 2 + 0 + 20 ) / 5 = 8 
    \end{description}
    \item[Process Execution Gantt Chart of non-preemptive Priority] \hfill 
    \begin{description}
        \item[Waiting Time of P1:] 17
        \item[Waiting Time of P2:] 0 
        \item[Waiting Time of P3:] 26
        \item[Waiting Time of P4:] 30
        \item[Waiting Time of P5:] 6
        \item[Average waiting time:] ( 17 + 0 + 26 + 30 + 6) / 5 = 15,8 
    \end{description} 
    
    \item[Process Execution Gantt Chart of RR] \hfill 
    \begin{description}
        \item[Waiting Time of P1:] 16 + 5 - 8 = 13
        \item[Waiting Time of P2:] 21 + 6 - 5 = 22
        \item[Waiting Time of P3:] 10
        \item[Waiting Time of P4:] 14 
        \item[Waiting Time of P5:] 27 + 8 -10 = 25
        \item[Average waiting time:] (13 + 22 + 10 + 14 + 25) / 5 = 16,8
    \end{description}
    
\end{description}

\subsection{Calculate average turnaround time for each of the scheduling algorithms in part (a).}
\begin{description}
    \item[Average turnaround time for FCFS] \hfill 
\\( 8 + 13 + 16 + 17 + 27 ) / 5 = 16.2
    \item[Average turnaround time for SJF] \hfill 
\\( 1 + 4 + 9 + 17 + 27 ) / 5 = 11.6   
    \item[Average turnaround time for non-preemptive Priority] \hfill 
\\( 5 + 15 + 23 + 26 + 27 ) / 5 = 19.2  
    \item[Average turnaround time for RR] \hfill 
\\( 20 + 21 + 11 + 12 + 27 ) / 5 = 18.2
\end{description}

\section{Problem 2}
Now assume that the context-switching overhead is equivalent to 0.5 ms. Calculate the CPU utilisation for all four scheduling algorithms in Problem 1.
\\ 
\begin{equation*}
    CPU utilisation  = 100 * \frac{{Total Time Spend on Processes}}{\mathrm{Total Time Spend on Execution}}
%\begin{center}\ce{}\end{center}
\end{equation*}
\begin{description}
    \item[CPU utilisation of FCFS] \hfill \\ 
(27 / 29 * 100) = 93,10
    \item[CPU utilisation of SJF] \hfill \\
    (27 / 29 * 100) = 93,10
    \item[CPU utilisation of non-preemptive Priority] \hfill \\
    (27 / 29 * 100) = 93,10
    \item[CPU utilisation of RR] \hfill \\
(27 / 31 * 100) = 87,09
\end{description}

\section{Problem 3}

\subsection{}
There is race condition in code, thus, output is non-deterministic and depends on order of process execution.
\\The available\_connections is shared variable and it might be written by multiple threads simultaneously inside of the connect() and disconnect() methods.
\\Since write is actually consist of multiple Instructions 
\subsection{}
We could use locks to prevent race condition.

\begin{lstlisting}[language=C]
//PROBLEM 3
#define MAX_CONNECTIONS 5000
int available_connections = MAX_CONNECTIONS;
mutex lock;

/* When a thread wishes to establish a connection with the rver,
it invokes the connect() function:*/

int connect() {
    mutex_lock(&lock); 
    if (available_connections < 1) {
        mutex_unlock(&lock);
        return -1;
    } else {
        available_connections--;
    }
    mutex_unlock(&lock);
    return 0;
}

 /* When a thread wishes to drop a connection with the server,
 it invokes disconnect() */
 int disconnect() {
    mutex_lock(&lock);
    available_connections++;
    mutex_unlock(&lock);
    return 0;
 }
\end{lstlisting}

\subsection{}To prevent race condition(s), could we replace the integer variable available connections
with atomic integer, which allows atomic update of a variable of this type? Explain your
answer.
\\ No, replacing the integer with atomic integer is not enough for preventing race condition. While one of the threads inside of if statement of connect (before returning) another thread might increase the available\_connections variable which might effect of the result of if statement.

\section{Problem 4}
We could use semaphore with initial value M ,however, if a party with more than 4 numcustomers arrive we need to be sure we have enough rooms. Thus, we can use a semaphore with initial value 1 and use it to check available number of rooms.
\begin{lstlisting}[language=C]

    semaphore room = 1;
    int roomNum = M;

    int makeReservation(int numCustomers) {
        wait(room);
        int roomNeeded = numCustomers / 4 + 1; 
        // int a = 2 / 4 + 1 = 1; a = 1
        if(roomNum >=roomNeeded ) {
            //accept
            roomNum = roomNum - roomNeeded;

        } else {
            //decline
        }
        signal(room);
        return 0;
    }

    int checkout(int roomNumber) {
        wait(room);
        roomNum = roomNum + roomNumber;
        signal(room);
    }

\end{lstlisting}
\newpage
\section{Problem 5}
\begin{lstlisting}[language=C]

    data d1;
    data d2;
    data d3;
    semaphore d1;
    semaphore d2;
    semaphore d3;
/* process P runs in this function */
void *P() {
    wait(d1);

    //makeCalculation

    signal(d1);
}
 
/* process Q runs in this function */
void *Q() {
    wait(d2);
    wait(d1);

    //makeCalculation

    signal(d1);
    signal(d2);
}

/* process R runs in this function */
void *R() {
    wait(d2);
    wait(d3);

    //makeCalculation

    signal(d3);
    signal(d2);
}

/* process S runs in this function */
void *S() {
    wait(d3);

    //makeCalculation

    signal(d3);
}
\end{lstlisting}
\end{document}